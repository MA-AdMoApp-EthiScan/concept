\section{Évaluations}

\subsection{Tests unitaires}

Afin de garantir la fiabilité et la robustesse de notre application mobile, nous avons mis en place une série de tests unitaires. Les tests unitaires sont essentiels pour vérifier que chaque composant de l'application fonctionne correctement de manière isolée.

Notre application utilise Firebase pour la gestion des données et l'authentification des utilisateurs, spécifiquement Firestore pour la base de données et Firebase Authentication pour la gestion des utilisateurs \ref{sec:architecture}. Étant donné que Firebase est une plateforme de services cloud, il est nécessaire d'utiliser des outils de mocking pour simuler les interactions avec ces services lors des tests unitaires ce qui a posé quelques difficultés \ref{sec:problems}.

Pour ce faire, nous avons intégré plusieurs librairies de mocking dans notre suite de tests :

\begin{itemize}[noitemsep]
    \item \textbf{Mockito} : Cette librairie nous permet de créer des objets mock et de définir le comportement attendu de ces objets lors des tests. Elle est utilisée pour simuler les interactions avec diverses dépendances de notre application.
    \item \textbf{MockFirebaseFirestore} : Cette librairie est utilisée pour simuler les opérations de la base de données Firestore. Elle permet de créer des collections et des documents fictifs, de simuler les requêtes et de vérifier les interactions avec Firestore sans nécessiter une connexion réelle à la base de données.
    \item \textbf{MockAuthentication} : Cette librairie est utilisée pour simuler les opérations d'authentification de Firebase. Elle permet de tester des scénarios d'inscription, de connexion et de déconnexion des utilisateurs sans interagir avec le service d'authentification réel.
\end{itemize}

\subsection{Tests utilisateurs}

Des tests utilisateurs ont été planifié mais n'ont pas pu être mis en oeuvre puisque nous avons préféré développer l'application entièrement avant de la faire tester. Il était prévu de réaliser une série de tests utilisateurs à des proches. Ces tests ont été conçus pour évaluer chaque user story et comprendre comment les utilisateurs interagissent avec notre application dans divers scénarios.

%En fin de compte, notre objectif était de créer une application qui répond aux besoins de nos utilisateurs de manière intuitive et efficace. Bien que nous n'ayons pas encore pu réaliser ces tests utilisateurs, ils restent une priorité pour notre équipe à mesure que nous continuons à développer et à améliorer notre application.
