\section{Techniques}

\subsection{Quoi et Comment}

EthiScan implémente un système de scan de code-barres qui identifie les produits et récupère des informations à partir de bases de données externes via des API. L'application traite ces données pour afficher les informations pertinentes aux utilisateurs, telles que les labels, l'évolution des prix, et l'impact carbone.

\subsection{Problèmes Rencontrés et Solutions}

Un défi a été l'optimisation du temps de réponse pour le scan de produits. Cela a été résolu en optimisant les requêtes de base de données et en utilisant un cache pour stocker les données des produits fréquemment scannés.

\subsection{Conclusion Technique}

La combinaison de Flutter et Firebase a prouvé son efficacité pour développer une application mobile performante et réactive. L'architecture choisie a permis une mise en œuvre rapide des fonctionnalités tout en maintenant une haute qualité et fiabilité des données.

\subsection{Auto-Critique du Code}

\subsubsection{Points Positifs}
\begin{itemize}
    \item Code bien structuré et commenté, facilitant la maintenance et les mises à jour.
    \item Utilisation efficace des patterns de conception pour une architecture solide.
\end{itemize}

\subsubsection{Points à Améliorer}
\begin{itemize}
    \item Couverture des tests unitaires à augmenter pour assurer une meilleure stabilité.
    \item Optimisation possible de certaines requêtes de données pour accélérer les temps de réponse.
\end{itemize}